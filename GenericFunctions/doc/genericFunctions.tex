\documentclass{report}
% Psfig/TeX 
\def\PsfigVersion{1.10}
\def\setDriver{\DvipsDriver} % \DvipsDriver or \OzTeXDriver
%
% All software, documentation, and related files in this distribution of
% psfig/tex are Copyright 1993 Trevor J. Darrell
%
% Permission is granted for use and non-profit distribution of psfig/tex 
% providing that this notice is clearly maintained. The right to
% distribute any portion of psfig/tex for profit or as part of any commercial
% product is specifically reserved for the author(s) of that portion.
%
% To use with LaTeX, use \documentstyle[psfig,...]{...}
% To use with TeX, use \input psfig.sty
%
% Bugs and improvements to trevor@media.mit.edu.
%
% Thanks to Ned Batchelder, Greg Hager (GDH), J. Daniel Smith (JDS),
% Tom Rokicki (TR), Robert Russell (RR), George V. Reilly (GVR),
% Ken McGlothlen (KHC), Baron Grey (BG), Gerhard Tobermann (GT).
% and all others who have contributed code and comments to this project!
%
% ======================================================================
% Modification History:
%
%  9 Oct 1990   JDS	used more robust bbox reading code from Tom Rokicki
% 29 Mar 1991   JDS	implemented rotation= option
% 25 Jun 1991   RR	if bb specified on cmd line don't check
%			for .ps file.
%  3 Jul 1991	JDS	check if file already read in once
%  4 Sep 1991	JDS	fixed incorrect computation of rotated
%			bounding box
% 25 Sep 1991	GVR	expanded synopsis of \psfig
% 14 Oct 1991	JDS	\fbox code from LaTeX so \psdraft works with TeX
%			changed \typeout to \ps@typeout
% 17 Oct 1991	JDS	added \psscalefirst and \psrotatefirst
% 23 Jun 1993   KHC     ``doclip'' must appear before ``rotate''
% 27 Oct 1993   TJD	removed printing of filename to avoid 
%			underscore problems. changed \frame to \fbox.
%			Added OzTeX support from BG. Added new
%			figure search path code from GT.
%
% ======================================================================
%
% Command synopsis:
%
% \psdraft	draws an outline box, but doesn't include the figure
%		in the DVI file.  Useful for previewing.
%
% \psfull	includes the figure in the DVI file (default).
%
% \psscalefirst width= or height= specifies the size of the figure
% 		before rotation.
% \psrotatefirst (default) width= or height= specifies the size of the
% 		 figure after rotation.  Asymetric figures will
% 		 appear to shrink.
%
% \psfigurepath{dir:dir:...}  sets the path to search for the figure
%
% \psfig
% usage: \psfig{file=, figure=, height=, width=,
%			bbllx=, bblly=, bburx=, bbury=,
%			rheight=, rwidth=, clip=, angle=, silent=}
%
%	"file" is the filename.  If no path name is specified and the
%		file is not found in the current directory,
%		it will be looked for in directory \psfigurepath.
%	"figure" is a synonym for "file".
%	By default, the width and height of the figure are taken from
%		the BoundingBox of the figure.
%	If "width" is specified, the figure is scaled so that it has
%		the specified width.  Its height changes proportionately.
%	If "height" is specified, the figure is scaled so that it has
%		the specified height.  Its width changes proportionately.
%	If both "width" and "height" are specified, the figure is scaled
%		anamorphically.
%	"bbllx", "bblly", "bburx", and "bbury" control the PostScript
%		BoundingBox.  If these four values are specified
%               *before* the "file" option, the PSFIG will not try to
%               open the PostScript file.
%	"rheight" and "rwidth" are the reserved height and width
%		of the figure, i.e., how big TeX actually thinks
%		the figure is.  They default to "width" and "height".
%	The "clip" option ensures that no portion of the figure will
%		appear outside its BoundingBox.  "clip=" is a switch and
%		takes no value, but the `=' must be present.
%	The "angle" option specifies the angle of rotation (degrees, ccw).
%	The "silent" option makes \psfig work silently.
%
% ======================================================================
% check to see if macros already loaded in (maybe some other file says
% "\input psfig") ...
\ifx\undefined\psfig\else\endinput\fi
%
% from a suggestion by eijkhout@csrd.uiuc.edu to allow
% loading as a style file. Changed to avoid problems
% with amstex per suggestion by jbence@math.ucla.edu

\let\LaTeXAtSign=\@
\let\@=\relax
\edef\psfigRestoreAt{\catcode`\@=\number\catcode`@\relax}
%\edef\psfigRestoreAt{\catcode`@=\number\catcode`@\relax}
\catcode`\@=11\relax
\newwrite\@unused
\def\ps@typeout#1{{\let\protect\string\immediate\write\@unused{#1}}}

\def\DvipsDriver{
	\ps@typeout{psfig/tex \PsfigVersion -dvips}
\def\PsfigSpecials{\DvipsSpecials} 	\def\ps@dir{/}
\def\ps@predir{} }
\def\OzTeXDriver{
	\ps@typeout{psfig/tex \PsfigVersion -oztex}
	\def\PsfigSpecials{\OzTeXSpecials}
	\def\ps@dir{:}
	\def\ps@predir{:}
	\catcode`\^^J=5
}

%% Here's how you define your figure path.  Should be set up with null
%% default and a user useable definition.

\def\figurepath{./:}
\def\psfigurepath#1{\edef\figurepath{#1:}}

%%% inserted for Searching Unixpaths
%%% (the path must end with :)
%%% (call: \DoPaths\figurepath )
%%%------------------------------------------------------
\def\DoPaths#1{\expandafter\EachPath#1\stoplist}
%
\def\leer{}
\def\EachPath#1:#2\stoplist{% #1 part of the list (delimiter :)
  \ExistsFile{#1}{\SearchedFile}
  \ifx#2\leer
  \else
    \expandafter\EachPath#2\stoplist
  \fi}
%
% exists the file (does not work for directories!)
%
\def\ps@dir{/}
\def\ExistsFile#1#2{%
   \openin1=\ps@predir#1\ps@dir#2
   \ifeof1
       \closein1
       %\ps@typeout{...not: \ps@predir#1\ps@dir#2}
   \else
       \closein1
       %\ps@typeout{...in:  \ps@predir#1\ps@dir#2}
        \ifx\ps@founddir\leer
          %\ps@typeout{set founddir #1}
           \edef\ps@founddir{#1}
        \fi
   \fi}
%------------------------------------------------------
%
% Get dir in path or error
%
\def\get@dir#1{%
  \def\ps@founddir{}
  \def\SearchedFile{#1}
  \DoPaths\figurepath
%  \fi
}
%------------------------------------------------------
%%% END of Searching Unixpaths


%
% @psdo control structure -- similar to Latex @for.
% I redefined these with different names so that psfig can
% be used with TeX as well as LaTeX, and so that it will not 
% be vunerable to future changes in LaTeX's internal
% control structure,
%
\def\@nnil{\@nil}
\def\@empty{}
\def\@psdonoop#1\@@#2#3{}
\def\@psdo#1:=#2\do#3{\edef\@psdotmp{#2}\ifx\@psdotmp\@empty \else
    \expandafter\@psdoloop#2,\@nil,\@nil\@@#1{#3}\fi}
\def\@psdoloop#1,#2,#3\@@#4#5{\def#4{#1}\ifx #4\@nnil \else
       #5\def#4{#2}\ifx #4\@nnil \else#5\@ipsdoloop #3\@@#4{#5}\fi\fi}
\def\@ipsdoloop#1,#2\@@#3#4{\def#3{#1}\ifx #3\@nnil 
       \let\@nextwhile=\@psdonoop \else
      #4\relax\let\@nextwhile=\@ipsdoloop\fi\@nextwhile#2\@@#3{#4}}
\def\@tpsdo#1:=#2\do#3{\xdef\@psdotmp{#2}\ifx\@psdotmp\@empty \else
    \@tpsdoloop#2\@nil\@nil\@@#1{#3}\fi}
\def\@tpsdoloop#1#2\@@#3#4{\def#3{#1}\ifx #3\@nnil 
       \let\@nextwhile=\@psdonoop \else
      #4\relax\let\@nextwhile=\@tpsdoloop\fi\@nextwhile#2\@@#3{#4}}
% 
% \fbox is defined in latex.tex; so if \fbox is undefined, assume that
% we are not in LaTeX.
% Perhaps this could be done better???
\ifx\undefined\fbox
% \fbox code from modified slightly from LaTeX
\newdimen\fboxrule
\newdimen\fboxsep
\newdimen\ps@tempdima
\newbox\ps@tempboxa
\fboxsep = 3pt
\fboxrule = .4pt
\long\def\fbox#1{\leavevmode\setbox\ps@tempboxa\hbox{#1}\ps@tempdima\fboxrule
    \advance\ps@tempdima \fboxsep \advance\ps@tempdima \dp\ps@tempboxa
   \hbox{\lower \ps@tempdima\hbox
  {\vbox{\hrule height \fboxrule
          \hbox{\vrule width \fboxrule \hskip\fboxsep
          \vbox{\vskip\fboxsep \box\ps@tempboxa\vskip\fboxsep}\hskip 
                 \fboxsep\vrule width \fboxrule}
                 \hrule height \fboxrule}}}}
\fi
%
%%%%%%%%%%%%%%%%%%%%%%%%%%%%%%%%%%%%%%%%%%%%%%%%%%%%%%%%%%%%%%%%%%%
% file reading stuff from epsf.tex
%   EPSF.TEX macro file:
%   Written by Tomas Rokicki of Radical Eye Software, 29 Mar 1989.
%   Revised by Don Knuth, 3 Jan 1990.
%   Revised by Tomas Rokicki to accept bounding boxes with no
%      space after the colon, 18 Jul 1990.
%   Portions modified/removed for use in PSFIG package by
%      J. Daniel Smith, 9 October 1990.
%
\newread\ps@stream
\newif\ifnot@eof       % continue looking for the bounding box?
\newif\if@noisy        % report what you're making?
\newif\if@atend        % %%BoundingBox: has (at end) specification
\newif\if@psfile       % does this look like a PostScript file?
%
% PostScript files should start with `%!'
%
{\catcode`\%=12\global\gdef\epsf@start{%!}}
\def\epsf@PS{PS}
%
\def\epsf@getbb#1{%
%
%   The first thing we need to do is to open the
%   PostScript file, if possible.
%
\openin\ps@stream=\ps@predir#1
\ifeof\ps@stream\ps@typeout{Error, File #1 not found}\else
%
%   Okay, we got it. Now we'll scan lines until we find one that doesn't
%   start with %. We're looking for the bounding box comment.
%
   {\not@eoftrue \chardef\other=12
    \def\do##1{\catcode`##1=\other}\dospecials \catcode`\ =10
    \loop
       \if@psfile
	  \read\ps@stream to \epsf@fileline
       \else{
	  \obeyspaces
          \read\ps@stream to \epsf@tmp\global\let\epsf@fileline\epsf@tmp}
       \fi
       \ifeof\ps@stream\not@eoffalse\else
%
%   Check the first line for `%!'.  Issue a warning message if its not
%   there, since the file might not be a PostScript file.
%
       \if@psfile\else
       \expandafter\epsf@test\epsf@fileline:. \\%
       \fi
%
%   We check to see if the first character is a % sign;
%   if so, we look further and stop only if the line begins with
%   `%%BoundingBox:' and the `(atend)' specification was not found.
%   That is, the only way to stop is when the end of file is reached,
%   or a `%%BoundingBox: llx lly urx ury' line is found.
%
          \expandafter\epsf@aux\epsf@fileline:. \\%
       \fi
   \ifnot@eof\repeat
   }\closein\ps@stream\fi}%
%
% This tests if the file we are reading looks like a PostScript file.
%
\long\def\epsf@test#1#2#3:#4\\{\def\epsf@testit{#1#2}
			\ifx\epsf@testit\epsf@start\else
\ps@typeout{Warning! File does not start with `\epsf@start'.  It may not be a PostScript file.}
			\fi
			\@psfiletrue} % don't test after 1st line
%
%   We still need to define the tricky \epsf@aux macro. This requires
%   a couple of magic constants for comparison purposes.
%
{\catcode`\%=12\global\let\epsf@percent=%\global\def\epsf@bblit{%BoundingBox}}
%
%
%   So we're ready to check for `%BoundingBox:' and to grab the
%   values if they are found.  We continue searching if `(at end)'
%   was found after the `%BoundingBox:'.
%
\long\def\epsf@aux#1#2:#3\\{\ifx#1\epsf@percent
   \def\epsf@testit{#2}\ifx\epsf@testit\epsf@bblit
	\@atendfalse
        \epsf@atend #3 . \\%
	\if@atend	
	   \if@verbose{
		\ps@typeout{psfig: found `(atend)'; continuing search}
	   }\fi
        \else
        \epsf@grab #3 . . . \\%
        \not@eoffalse
        \global\no@bbfalse
        \fi
   \fi\fi}%
%
%   Here we grab the values and stuff them in the appropriate definitions.
%
\def\epsf@grab #1 #2 #3 #4 #5\\{%
   \global\def\epsf@llx{#1}\ifx\epsf@llx\empty
      \epsf@grab #2 #3 #4 #5 .\\\else
   \global\def\epsf@lly{#2}%
   \global\def\epsf@urx{#3}\global\def\epsf@ury{#4}\fi}%
%
% Determine if the stuff following the %%BoundingBox is `(atend)'
% J. Daniel Smith.  Copied from \epsf@grab above.
%
\def\epsf@atendlit{(atend)} 
\def\epsf@atend #1 #2 #3\\{%
   \def\epsf@tmp{#1}\ifx\epsf@tmp\empty
      \epsf@atend #2 #3 .\\\else
   \ifx\epsf@tmp\epsf@atendlit\@atendtrue\fi\fi}


% End of file reading stuff from epsf.tex
%%%%%%%%%%%%%%%%%%%%%%%%%%%%%%%%%%%%%%%%%%%%%%%%%%%%%%%%%%%%%%%%%%%

%%%%%%%%%%%%%%%%%%%%%%%%%%%%%%%%%%%%%%%%%%%%%%%%%%%%%%%%%%%%%%%%%%%
% trigonometry stuff from "trig.tex"
\chardef\psletter = 11 % won't conflict with \begin{letter} now...
\chardef\other = 12

\newif \ifdebug %%% turn me on to see TeX hard at work ...
\newif\ifc@mpute %%% don't need to compute some values
\c@mputetrue % but assume that we do

\let\then = \relax
\def\r@dian{pt }
\let\r@dians = \r@dian
\let\dimensionless@nit = \r@dian
\let\dimensionless@nits = \dimensionless@nit
\def\internal@nit{sp }
\let\internal@nits = \internal@nit
\newif\ifstillc@nverging
\def \Mess@ge #1{\ifdebug \then \message {#1} \fi}

{ %%% Things that need abnormal catcodes %%%
	\catcode `\@ = \psletter
	\gdef \nodimen {\expandafter \n@dimen \the \dimen}
	\gdef \term #1 #2 #3%
	       {\edef \t@ {\the #1}%%% freeze parameter 1 (count, by value)
		\edef \t@@ {\expandafter \n@dimen \the #2\r@dian}%
				   %%% freeze parameter 2 (dimen, by value)
		\t@rm {\t@} {\t@@} {#3}%
	       }
	\gdef \t@rm #1 #2 #3%
	       {{%
		\count 0 = 0
		\dimen 0 = 1 \dimensionless@nit
		\dimen 2 = #2\relax
		\Mess@ge {Calculating term #1 of \nodimen 2}%
		\loop
		\ifnum	\count 0 < #1
		\then	\advance \count 0 by 1
			\Mess@ge {Iteration \the \count 0 \space}%
			\Multiply \dimen 0 by {\dimen 2}%
			\Mess@ge {After multiplication, term = \nodimen 0}%
			\Divide \dimen 0 by {\count 0}%
			\Mess@ge {After division, term = \nodimen 0}%
		\repeat
		\Mess@ge {Final value for term #1 of 
				\nodimen 2 \space is \nodimen 0}%
		\xdef \Term {#3 = \nodimen 0 \r@dians}%
		\aftergroup \Term
	       }}
	\catcode `\p = \other
	\catcode `\t = \other
	\gdef \n@dimen #1pt{#1} %%% throw away the ``pt''
}

\def \Divide #1by #2{\divide #1 by #2} %%% just a synonym

\def \Multiply #1by #2%%% allows division of a dimen by a dimen
       {{%%% should really freeze parameter 2 (dimen, passed by value)
	\count 0 = #1\relax
	\count 2 = #2\relax
	\count 4 = 65536
	\Mess@ge {Before scaling, count 0 = \the \count 0 \space and
			count 2 = \the \count 2}%
	\ifnum	\count 0 > 32767 %%% do our best to avoid overflow
	\then	\divide \count 0 by 4
		\divide \count 4 by 4
	\else	\ifnum	\count 0 < -32767
		\then	\divide \count 0 by 4
			\divide \count 4 by 4
		\else
		\fi
	\fi
	\ifnum	\count 2 > 32767 %%% while retaining reasonable accuracy
	\then	\divide \count 2 by 4
		\divide \count 4 by 4
	\else	\ifnum	\count 2 < -32767
		\then	\divide \count 2 by 4
			\divide \count 4 by 4
		\else
		\fi
	\fi
	\multiply \count 0 by \count 2
	\divide \count 0 by \count 4
	\xdef \product {#1 = \the \count 0 \internal@nits}%
	\aftergroup \product
       }}

\def\r@duce{\ifdim\dimen0 > 90\r@dian \then   % sin(x+90) = sin(180-x)
		\multiply\dimen0 by -1
		\advance\dimen0 by 180\r@dian
		\r@duce
	    \else \ifdim\dimen0 < -90\r@dian \then  % sin(-x) = sin(360+x)
		\advance\dimen0 by 360\r@dian
		\r@duce
		\fi
	    \fi}

\def\Sine#1%
       {{%
	\dimen 0 = #1 \r@dian
	\r@duce
	\ifdim\dimen0 = -90\r@dian \then
	   \dimen4 = -1\r@dian
	   \c@mputefalse
	\fi
	\ifdim\dimen0 = 90\r@dian \then
	   \dimen4 = 1\r@dian
	   \c@mputefalse
	\fi
	\ifdim\dimen0 = 0\r@dian \then
	   \dimen4 = 0\r@dian
	   \c@mputefalse
	\fi
%
	\ifc@mpute \then
        	% convert degrees to radians
		\divide\dimen0 by 180
		\dimen0=3.141592654\dimen0
%
		\dimen 2 = 3.1415926535897963\r@dian %%% a well-known constant
		\divide\dimen 2 by 2 %%% we only deal with -pi/2 : pi/2
		\Mess@ge {Sin: calculating Sin of \nodimen 0}%
		\count 0 = 1 %%% see power-series expansion for sine
		\dimen 2 = 1 \r@dian %%% ditto
		\dimen 4 = 0 \r@dian %%% ditto
		\loop
			\ifnum	\dimen 2 = 0 %%% then we've done
			\then	\stillc@nvergingfalse 
			\else	\stillc@nvergingtrue
			\fi
			\ifstillc@nverging %%% then calculate next term
			\then	\term {\count 0} {\dimen 0} {\dimen 2}%
				\advance \count 0 by 2
				\count 2 = \count 0
				\divide \count 2 by 2
				\ifodd	\count 2 %%% signs alternate
				\then	\advance \dimen 4 by \dimen 2
				\else	\advance \dimen 4 by -\dimen 2
				\fi
		\repeat
	\fi		
			\xdef \sine {\nodimen 4}%
       }}

% Now the Cosine can be calculated easily by calling \Sine
\def\Cosine#1{\ifx\sine\UnDefined\edef\Savesine{\relax}\else
		             \edef\Savesine{\sine}\fi
	{\dimen0=#1\r@dian\advance\dimen0 by 90\r@dian
	 \Sine{\nodimen 0}
	 \xdef\cosine{\sine}
	 \xdef\sine{\Savesine}}}	      
% end of trig stuff
%%%%%%%%%%%%%%%%%%%%%%%%%%%%%%%%%%%%%%%%%%%%%%%%%%%%%%%%%%%%%%%%%%%%

\def\psdraft{
	\def\@psdraft{0}
	%\ps@typeout{draft level now is \@psdraft \space . }
}
\def\psfull{
	\def\@psdraft{100}
	%\ps@typeout{draft level now is \@psdraft \space . }
}

\psfull

\newif\if@scalefirst
\def\psscalefirst{\@scalefirsttrue}
\def\psrotatefirst{\@scalefirstfalse}
\psrotatefirst

\newif\if@draftbox
\def\psnodraftbox{
	\@draftboxfalse
}
\def\psdraftbox{
	\@draftboxtrue
}
\@draftboxtrue

\newif\if@prologfile
\newif\if@postlogfile
\def\pssilent{
	\@noisyfalse
}
\def\psnoisy{
	\@noisytrue
}
\psnoisy
%%% These are for the option list.
%%% A specification of the form a = b maps to calling \@p@@sa{b}
\newif\if@bbllx
\newif\if@bblly
\newif\if@bburx
\newif\if@bbury
\newif\if@height
\newif\if@width
\newif\if@rheight
\newif\if@rwidth
\newif\if@angle
\newif\if@clip
\newif\if@verbose
\def\@p@@sclip#1{\@cliptrue}
%
%
\newif\if@decmpr
%
\def\@p@@sfigure#1{\def\@p@sfile{null}\def\@p@sbbfile{null}\@decmprfalse
   % look directly for file (e.g. absolute path)
   \openin1=\ps@predir#1
   \ifeof1
	\closein1
	% failed, search directories for file
	\get@dir{#1}
	\ifx\ps@founddir\leer
		% failed, search directly for file.bb
		\openin1=\ps@predir#1.bb
		\ifeof1
			\closein1
			% failed, search directories for file.bb
			\get@dir{#1.bb}
			\ifx\ps@founddir\leer
				% failed, lose.
				\ps@typeout{Can't find #1 in \figurepath}
			\else
				% found file.bb in search dir
				\@decmprtrue
				\def\@p@sfile{\ps@founddir\ps@dir#1}
				\def\@p@sbbfile{\ps@founddir\ps@dir#1.bb}
			\fi
		\else
			\closein1
			%found file.bb directly
			\@decmprtrue
			\def\@p@sfile{#1}
			\def\@p@sbbfile{#1.bb}
		\fi
	\else
		% found file in search dir
		\def\@p@sfile{\ps@founddir\ps@dir#1}
		\def\@p@sbbfile{\ps@founddir\ps@dir#1}
	\fi
   \else
	% found file directly
	\closein1
	\def\@p@sfile{#1}
	\def\@p@sbbfile{#1}
   \fi
}
%
%
%
\def\@p@@sfile#1{\@p@@sfigure{#1}}
%
\def\@p@@sbbllx#1{
		%\ps@typeout{bbllx is #1}
		\@bbllxtrue
		\dimen100=#1
		\edef\@p@sbbllx{\number\dimen100}
}
\def\@p@@sbblly#1{
		%\ps@typeout{bblly is #1}
		\@bbllytrue
		\dimen100=#1
		\edef\@p@sbblly{\number\dimen100}
}
\def\@p@@sbburx#1{
		%\ps@typeout{bburx is #1}
		\@bburxtrue
		\dimen100=#1
		\edef\@p@sbburx{\number\dimen100}
}
\def\@p@@sbbury#1{
		%\ps@typeout{bbury is #1}
		\@bburytrue
		\dimen100=#1
		\edef\@p@sbbury{\number\dimen100}
}
\def\@p@@sheight#1{
		\@heighttrue
		\dimen100=#1
   		\edef\@p@sheight{\number\dimen100}
		%\ps@typeout{Height is \@p@sheight}
}
\def\@p@@swidth#1{
		%\ps@typeout{Width is #1}
		\@widthtrue
		\dimen100=#1
		\edef\@p@swidth{\number\dimen100}
}
\def\@p@@srheight#1{
		%\ps@typeout{Reserved height is #1}
		\@rheighttrue
		\dimen100=#1
		\edef\@p@srheight{\number\dimen100}
}
\def\@p@@srwidth#1{
		%\ps@typeout{Reserved width is #1}
		\@rwidthtrue
		\dimen100=#1
		\edef\@p@srwidth{\number\dimen100}
}
\def\@p@@sangle#1{
		%\ps@typeout{Rotation is #1}
		\@angletrue
%		\dimen100=#1
		\edef\@p@sangle{#1} %\number\dimen100}
}
\def\@p@@ssilent#1{ 
		\@verbosefalse
}
\def\@p@@sprolog#1{\@prologfiletrue\def\@prologfileval{#1}}
\def\@p@@spostlog#1{\@postlogfiletrue\def\@postlogfileval{#1}}
\def\@cs@name#1{\csname #1\endcsname}
\def\@setparms#1=#2,{\@cs@name{@p@@s#1}{#2}}
%
% initialize the defaults (size the size of the figure)
%
\def\ps@init@parms{
		\@bbllxfalse \@bbllyfalse
		\@bburxfalse \@bburyfalse
		\@heightfalse \@widthfalse
		\@rheightfalse \@rwidthfalse
		\def\@p@sbbllx{}\def\@p@sbblly{}
		\def\@p@sbburx{}\def\@p@sbbury{}
		\def\@p@sheight{}\def\@p@swidth{}
		\def\@p@srheight{}\def\@p@srwidth{}
		\def\@p@sangle{0}
		\def\@p@sfile{} \def\@p@sbbfile{}
		\def\@p@scost{10}
		\def\@sc{}
		\@prologfilefalse
		\@postlogfilefalse
		\@clipfalse
		\if@noisy
			\@verbosetrue
		\else
			\@verbosefalse
		\fi
}
%
% Go through the options setting things up.
%
\def\parse@ps@parms#1{
	 	\@psdo\@psfiga:=#1\do
		   {\expandafter\@setparms\@psfiga,}}
%
% Compute bb height and width
%
\newif\ifno@bb
\def\bb@missing{
	\if@verbose{
		\ps@typeout{psfig: searching \@p@sbbfile \space  for bounding box}
	}\fi
	\no@bbtrue
	\epsf@getbb{\@p@sbbfile}
        \ifno@bb \else \bb@cull\epsf@llx\epsf@lly\epsf@urx\epsf@ury\fi
}	
\def\bb@cull#1#2#3#4{
	\dimen100=#1 bp\edef\@p@sbbllx{\number\dimen100}
	\dimen100=#2 bp\edef\@p@sbblly{\number\dimen100}
	\dimen100=#3 bp\edef\@p@sbburx{\number\dimen100}
	\dimen100=#4 bp\edef\@p@sbbury{\number\dimen100}
	\no@bbfalse
}
% rotate point (#1,#2) about (0,0).
% The sine and cosine of the angle are already stored in \sine and
% \cosine.  The result is placed in (\p@intvaluex, \p@intvaluey).
\newdimen\p@intvaluex
\newdimen\p@intvaluey
\def\rotate@#1#2{{\dimen0=#1 sp\dimen1=#2 sp
%            	calculate x' = x \cos\theta - y \sin\theta
		  \global\p@intvaluex=\cosine\dimen0
		  \dimen3=\sine\dimen1
		  \global\advance\p@intvaluex by -\dimen3
% 		calculate y' = x \sin\theta + y \cos\theta
		  \global\p@intvaluey=\sine\dimen0
		  \dimen3=\cosine\dimen1
		  \global\advance\p@intvaluey by \dimen3
		  }}
\def\compute@bb{
		\no@bbfalse
		\if@bbllx \else \no@bbtrue \fi
		\if@bblly \else \no@bbtrue \fi
		\if@bburx \else \no@bbtrue \fi
		\if@bbury \else \no@bbtrue \fi
		\ifno@bb \bb@missing \fi
		\ifno@bb \ps@typeout{FATAL ERROR: no bb supplied or found}
			\no-bb-error
		\fi
		%
%\ps@typeout{BB: \@p@sbbllx, \@p@sbblly, \@p@sbburx, \@p@sbbury} 
%
% store height/width of original (unrotated) bounding box
		\count203=\@p@sbburx
		\count204=\@p@sbbury
		\advance\count203 by -\@p@sbbllx
		\advance\count204 by -\@p@sbblly
		\edef\ps@bbw{\number\count203}
		\edef\ps@bbh{\number\count204}
		%\ps@typeout{ psbbh = \ps@bbh, psbbw = \ps@bbw }
		\if@angle 
			\Sine{\@p@sangle}\Cosine{\@p@sangle}
	        	{\dimen100=\maxdimen\xdef\r@p@sbbllx{\number\dimen100}
					    \xdef\r@p@sbblly{\number\dimen100}
			                    \xdef\r@p@sbburx{-\number\dimen100}
					    \xdef\r@p@sbbury{-\number\dimen100}}
%
% Need to rotate all four points and take the X-Y extremes of the new
% points as the new bounding box.
                        \def\minmaxtest{
			   \ifnum\number\p@intvaluex<\r@p@sbbllx
			      \xdef\r@p@sbbllx{\number\p@intvaluex}\fi
			   \ifnum\number\p@intvaluex>\r@p@sbburx
			      \xdef\r@p@sbburx{\number\p@intvaluex}\fi
			   \ifnum\number\p@intvaluey<\r@p@sbblly
			      \xdef\r@p@sbblly{\number\p@intvaluey}\fi
			   \ifnum\number\p@intvaluey>\r@p@sbbury
			      \xdef\r@p@sbbury{\number\p@intvaluey}\fi
			   }
%			lower left
			\rotate@{\@p@sbbllx}{\@p@sbblly}
			\minmaxtest
%			upper left
			\rotate@{\@p@sbbllx}{\@p@sbbury}
			\minmaxtest
%			lower right
			\rotate@{\@p@sbburx}{\@p@sbblly}
			\minmaxtest
%			upper right
			\rotate@{\@p@sbburx}{\@p@sbbury}
			\minmaxtest
			\edef\@p@sbbllx{\r@p@sbbllx}\edef\@p@sbblly{\r@p@sbblly}
			\edef\@p@sbburx{\r@p@sbburx}\edef\@p@sbbury{\r@p@sbbury}
%\ps@typeout{rotated BB: \r@p@sbbllx, \r@p@sbblly, \r@p@sbburx, \r@p@sbbury}
		\fi
		\count203=\@p@sbburx
		\count204=\@p@sbbury
		\advance\count203 by -\@p@sbbllx
		\advance\count204 by -\@p@sbblly
		\edef\@bbw{\number\count203}
		\edef\@bbh{\number\count204}
		%\ps@typeout{ bbh = \@bbh, bbw = \@bbw }
}
%
% \in@hundreds performs #1 * (#2 / #3) correct to the hundreds,
%	then leaves the result in @result
%
\def\in@hundreds#1#2#3{\count240=#2 \count241=#3
		     \count100=\count240	% 100 is first digit #2/#3
		     \divide\count100 by \count241
		     \count101=\count100
		     \multiply\count101 by \count241
		     \advance\count240 by -\count101
		     \multiply\count240 by 10
		     \count101=\count240	%101 is second digit of #2/#3
		     \divide\count101 by \count241
		     \count102=\count101
		     \multiply\count102 by \count241
		     \advance\count240 by -\count102
		     \multiply\count240 by 10
		     \count102=\count240	% 102 is the third digit
		     \divide\count102 by \count241
		     \count200=#1\count205=0
		     \count201=\count200
			\multiply\count201 by \count100
		 	\advance\count205 by \count201
		     \count201=\count200
			\divide\count201 by 10
			\multiply\count201 by \count101
			\advance\count205 by \count201
			%
		     \count201=\count200
			\divide\count201 by 100
			\multiply\count201 by \count102
			\advance\count205 by \count201
			%
		     \edef\@result{\number\count205}
}
\def\compute@wfromh{
		% computing : width = height * (bbw / bbh)
		\in@hundreds{\@p@sheight}{\@bbw}{\@bbh}
		%\ps@typeout{ \@p@sheight * \@bbw / \@bbh, = \@result }
		\edef\@p@swidth{\@result}
		%\ps@typeout{w from h: width is \@p@swidth}
}
\def\compute@hfromw{
		% computing : height = width * (bbh / bbw)
	        \in@hundreds{\@p@swidth}{\@bbh}{\@bbw}
		%\ps@typeout{ \@p@swidth * \@bbh / \@bbw = \@result }
		\edef\@p@sheight{\@result}
		%\ps@typeout{h from w : height is \@p@sheight}
}
\def\compute@handw{
		\if@height 
			\if@width
			\else
				\compute@wfromh
			\fi
		\else 
			\if@width
				\compute@hfromw
			\else
				\edef\@p@sheight{\@bbh}
				\edef\@p@swidth{\@bbw}
			\fi
		\fi
}
\def\compute@resv{
		\if@rheight \else \edef\@p@srheight{\@p@sheight} \fi
		\if@rwidth \else \edef\@p@srwidth{\@p@swidth} \fi
		%\ps@typeout{rheight = \@p@srheight, rwidth = \@p@srwidth}
}
%		
% Compute any missing values
\def\compute@sizes{
	\compute@bb
	\if@scalefirst\if@angle
% at this point the bounding box has been adjsuted correctly for
% rotation.  PSFIG does all of its scaling using \@bbh and \@bbw.  If
% a width= or height= was specified along with \psscalefirst, then the
% width=/height= value needs to be adjusted to match the new (rotated)
% bounding box size (specifed in \@bbw and \@bbh).
%    \ps@bbw       width=
%    -------  =  ---------- 
%    \@bbw       new width=
% so `new width=' = (width= * \@bbw) / \ps@bbw; where \ps@bbw is the
% width of the original (unrotated) bounding box.
	\if@width
	   \in@hundreds{\@p@swidth}{\@bbw}{\ps@bbw}
	   \edef\@p@swidth{\@result}
	\fi
	\if@height
	   \in@hundreds{\@p@sheight}{\@bbh}{\ps@bbh}
	   \edef\@p@sheight{\@result}
	\fi
	\fi\fi
	\compute@handw
	\compute@resv}
%
%
%
\def\OzTeXSpecials{
	\special{empty.ps /@isp {true} def}
	\special{empty.ps \@p@swidth \space \@p@sheight \space
			\@p@sbbllx \space \@p@sbblly \space
			\@p@sbburx \space \@p@sbbury \space
			startTexFig \space }
	\if@clip{
		\if@verbose{
			\ps@typeout{(clip)}
		}\fi
		\special{empty.ps doclip \space }
	}\fi
	\if@angle{
		\if@verbose{
			\ps@typeout{(rotate)}
		}\fi
		\special {empty.ps \@p@sangle \space rotate \space} 
	}\fi
	\if@prologfile
	    \special{\@prologfileval \space } \fi
	\if@decmpr{
		\if@verbose{
			\ps@typeout{psfig: Compression not available
			in OzTeX version \space }
		}\fi
	}\else{
		\if@verbose{
			\ps@typeout{psfig: including \@p@sfile \space }
		}\fi
		\special{epsf=\@p@sfile \space }
	}\fi
	\if@postlogfile
	    \special{\@postlogfileval \space } \fi
	\special{empty.ps /@isp {false} def}
}
\def\DvipsSpecials{
	%
	\special{ps::[begin] 	\@p@swidth \space \@p@sheight \space
			\@p@sbbllx \space \@p@sbblly \space
			\@p@sbburx \space \@p@sbbury \space
			startTexFig \space }
	\if@clip{
		\if@verbose{
			\ps@typeout{(clip)}
		}\fi
		\special{ps:: doclip \space }
	}\fi
	\if@angle
		\if@verbose{
			\ps@typeout{(clip)}
		}\fi
		\special {ps:: \@p@sangle \space rotate \space} 
	\fi
	\if@prologfile
	    \special{ps: plotfile \@prologfileval \space } \fi
	\if@decmpr{
		\if@verbose{
			\ps@typeout{psfig: including \@p@sfile.Z \space }
		}\fi
		\special{ps: plotfile "`zcat \@p@sfile.Z" \space }
	}\else{
		\if@verbose{
			\ps@typeout{psfig: including \@p@sfile \space }
		}\fi
		\special{ps: plotfile \@p@sfile \space }
	}\fi
	\if@postlogfile
	    \special{ps: plotfile \@postlogfileval \space } \fi
	\special{ps::[end] endTexFig \space }
}
%
% \psfig
% usage : \psfig{file=, height=, width=, bbllx=, bblly=, bburx=, bbury=,
%			rheight=, rwidth=, clip=}
%
% "clip=" is a switch and takes no value, but the `=' must be present.
\def\psfig#1{\vbox {
	% do a zero width hard space so that a single
	% \psfig in a centering enviornment will behave nicely
	%{\setbox0=\hbox{\ }\ \hskip-\wd0}
	%
	\ps@init@parms
	\parse@ps@parms{#1}
	\compute@sizes
	%
	\ifnum\@p@scost<\@psdraft{
		\PsfigSpecials 
		% Create the vbox to reserve the space for the figure.
		\vbox to \@p@srheight sp{
		% 1/92 TJD Changed from "true sp" to "sp" for magnification.
			\hbox to \@p@srwidth sp{
				\hss
			}
		\vss
		}
	}\else{
		% draft figure, just reserve the space and print the
		% path name.
		\if@draftbox{		
			% Verbose draft: print file name in box
			% 10/93 TJD changed to fbox from frame
			\hbox{\fbox{\vbox to \@p@srheight sp{
			\vss
			\hbox to \@p@srwidth sp{ \hss 
			        % 10/93 TJD deleted to avoid ``_'' problems
				% \@p@sfile
			 \hss }
			\vss
			}}}
		}\else{
			% Non-verbose draft
			\vbox to \@p@srheight sp{
			\vss
			\hbox to \@p@srwidth sp{\hss}
			\vss
			}
		}\fi	



	}\fi
}}
\psfigRestoreAt
\setDriver
\let\@=\LaTeXAtSign




\oddsidemargin 0.0in
\evensidemargin 0.0in
\setlength{\unitlength}{1mm}
\textheight 8.75in
\textwidth 6.5in
\parskip 1.45ex
\begin{document}

\title{A Library of Function Objects}
\author{J. Boudreau, Mark Fischler, Petar Maksimovic}
\maketitle

\section{Why Function Objects?}

\noindent In many applications it is desireable to treat mathematical functions
as objects; the action of function-objects on their arguments and on
each other (in other words, their algebra) can be defined in C++ so
that it reflects the acutal mathematics, and instances of thes
functions can be applied flexibly at run time either to data
representing arugments or to other functions.  Well-known use cases
include: plotting, data modelling, simulation, function approximation,
and integral transforms.

\noindent Using pointers-to-functions is a frequently used approach in either C or
FORTRAN which gives some run-time flexibility but not nearly the power of
function-objects.  Using the native C math library we can write 
\verb#sin(x)+exp(x)#,  but we cannot write \verb#sin+exp# nor pass this 
sum to other procedures.  However, since we are free to overload operators
in C++, we can get around this shortcoming by endowing the abstract interface
to all function classes with all the operations we want our functions to
have.  For clarity we have attempted to restrict this interface to 
mathematically well defined operations, which will be discussed below.  Another
way to view this design is as one in which the abstract interface to functions
permits the function library to be extended {\it not only through subclassing},
but also through ``composition''\footnote{here we mean composition in the
sense of obect composition not function composition}. We believe that the 
composition of functions through arithmetic operations is simple and
intuitive since it is based on algebraic rules we've learned during childhood
and is expressed in the same natural language.

\noindent In addition, we want to control the shape of a function: when we fit a 
function to data, for example.  We can do so by possibly associating one or 
more {\it parameters} to a function, such as amplitude or frequency, lifetime,
or width.  This can be accomplished in C++ with parameter objects that can
be part of or composed together with the functions.  Altering a parameter 
alters the function or functions of which it is a component.

\noindent We have written a small class library (``GenericFunctions'') which implements
 the features described above, for inclusion in the CLHEP project.  
Although our class library does not contain a comprehensive set of functions
for mathematics and physics, it does provide an extensible framework for 
developing such a library.  At the present writing it contains:
\begin{itemize}
\item{An abstract base class for functions.}
\item{Classes representing parameters.}
\item{Arithmetic operations acting on both functions and parameters}
\item{Class representing a possibly multidimensional {\it argument} to 
      a function}	
\item{A small set of implemented functions}
\end{itemize}

\section{A Word to the Wise}
\noindent Somebody told you once that C++ software is self-documenting.  Being generally
a trusting individual, you believed this for a while.  But now, you're not so 
naive.

\noindent You're going to need documentation in order to make sense of the
Generic Functions library.  {\it This} is the documentation you'll need.  
There are a large number of classes here that you as a user don't need to
know about at all.  The header files aren't encrypted so you can browse
them if you like, but you won't learn much that way.  Read this documention
instead.  Thank you.

\section{Example Application} 
\noindent Our example application is a program to demonstrate the phenomena of 
interference and diffraction.  This standalone program should allow one
to control the width and separation of two slits in a filter, and also
the intensity of light from each of the slits.  As we change the parameters
describing these variables we wish to see the {\it impulse function}, or
the intensity of light radiation at the position of the filter, change
in real time.  Also we wish to simultaneously see the {\it response function}
or the intensity pattern on the far screen, change in a way that is controlled
by the same parameters.  For this example, we are not going to worry about
how to display the function.  Graphics are outside the scope of the Generic 
Functions library.  However just imagine that there is a plotter somewhere
that gets a function object \verb#f# and invokes the function-call operator
during plotting, like this:
\begin{verbatim}
  double y = f(x);  // f is a function object
\end{verbatim}

\noindent The construction of functions is more involved than their invocation, so we're
going to look at the code that sets up the functions and ties their shapes
to the four parameters listed above.  This code is shown in 
Example~\ref{ref:ExampleCode}, while screen shots from an application are shown
in Figs.~\ref{ref:WideOpen}, \ref{ref:PartiallyClosed}, and \ref{ref:TotallyClosed}.

\noindent Fig.~\ref{ref:WideOpen} shows both slits wide open and the classic double-slit
interference pattern on the screen.  Fig.~\ref{ref:PartiallyClosed} shows the 
the pattern when one of the slits is partially closed and the interference
fringes are less sharp, and Fig~\ref{ref:TotallyClosed} shows the one of the
slits fully closed.  In this last case you can see that the interference pattern
has turned into a single-slit diffraction pattern.

\noindent We have two functions that need to be displayed:  the impulse function and the
response function.  Neither of these functions are part of the library {\it per
se}, but we can build them both out of the primitive functions \verb#Rectangular#
(for the impulse function) and \verb#Sin# and \verb#Cos# (for the response
function).  which are in the library.  The response function, by the way, is
given by the following expression:
\begin{displaymath}
	I = \left[ A_0 \sin{ax/2}/(ax/2) \right ]^2 + 
	    \left[ A_1 \sin{ax/2}/(ax/2) \right ]^2 +
	    2A_0A_1\left[\sin{ax/2}/(ax/2) \right ]^2 \cos{dx}
\end{displaymath}
where $x=\sin{\theta}$ and $a$ is equal to the slit widht in units of the wavelength,
$d$ is equal to the separation between the slits in units of the wavelength, and $A_0$
and $A_1$ are the amplitudes from the two slits.  The functions we require are simple 
enough to be built easily but complicated enough to illustrate several fundamental
features of the library.
 
\begin{figure}
\centerline{\makebox{\psfig{figure=example.ps}}}
\caption{Example.  Use of Generic Functions libary.  See text for explanation.}
\label{ref:ExampleCode}
\end{figure}
 
   
\begin{figure}
\centerline{\makebox{\psfig{figure=WideOpen.ps}}}
\caption{Picture of the example application which is discussed in the text.
Above, the impulse function shows both slots wide open. Each of the sliders
changes parameters and causes the plotter to update.  Below, the response function
shows the classic two-slit interference pattern.}
\label{ref:WideOpen}
\end{figure}

\begin{figure}
\centerline{\makebox{\psfig{figure=PartiallyClosed.ps}}}
\caption{The parameter values are now changed.  All of the functions, both primitive
and derived, change their shape in response.  Not that changing one parameter has
affected both functions.}
\label{ref:PartiallyClosed}
\end{figure}

\begin{figure}
\centerline{\makebox{\psfig{figure=TotallyClosed.ps}}}
\caption{The second slit has been effectively closed, by setting the intensity
of light through this slit to zero.  The two-slit interference pattern then
reduces to single-slit diffraction.  This simple classroom demonstration program 
can now be used as an aid in discussing optics or quantum mechanics, and can be
built in a few lines with a few minutes of programmer's time.}
\label{ref:TotallyClosed}
\end{figure}

\noindent The basic parameters of the model are the intensities of the two slits, the width
of the slits (this program does not allow the two slit widths to be varied independently)
and the separation.  These parameters are set up in lines 1-4 of the example.  The 
variables corresponding to these parameters are called \verb#a1#, \verb#a2#, \verb#s#, 
and \verb#d#.  

\noindent However some parameters of the impulse and response functions do not conveniently map
onto these parameters but to simple combinations thereof.  So, we can make derived parameters
out of the basic input parameters.  A derived parameter is a \verb#GENPARAMETER#.  This is
shown in lines 5-8 where derived parameters \verb#x0_0#, \verb#x0_1#, \verb#x1_0#, 
\verb#x1_1# are defined in terms of input parameters.

\noindent The impulse function will be built out of two rectangular functions.  So, we instantiate
these functions (Line 9), connect their internal parameters to the input parameters (Line 10-11)
and to the derived parameters (Lines 12-15).  The input parameters are now referenced both
by the derived parameters and by functions and {\bf must not go out of scope} until the
functions and derived parameters are no longer needed.

\noindent Now, whenever we vary the external parameter we're going to change the shape of the function.
The two rectangular functions can be added to obtain the response function (Line 16).  The sum of
the two functions maintains its connections to the controlling parameters.  The four parameters
with variable names  \verb#a1#, \verb#a2#, \verb#s#, and \verb#d# now not only control the 
two rectangular functions, but also their sum.

\noindent In the next few lines we build the response function, which is
somewhat more involved.  First we make instances of the functions
we're going to use (Lines 17-20).  Among these functions is a function
\verb#x# of class \verb#Variable#, which just returns the value of its
argument.  We multiply this function by the constant 0.5 and by the
Parameter \verb#s# to obtain a new function, \verb#alpha# in Line 21, and then
perform a similar set of operations to obtain a function
\verb#beta# in Line 22.  A derived function is a GENFUNCTION.   Our algebra is 
defined on all of these data types. We also use functions of type
\verb#Square#, \verb#Sin#, and \verb#Cos#.

\noindent The composition of functions is indicated by parenthensis \verb#()#.  The function \verb#square#
acts by squaring its argument.  The function \verb#amplitude0# is a simple function of the
angular distance \verb#x#. So, the function which we naturally express as \verb#square(amplitude0)#
in Line 23, is the composition of the function \verb#square# with the function \verb#amplitude0#.  In 
this example, function composition is used again in lines 24-28.

\noindent In general, we have tried to design this library so that single dimensional functions, 
multidimensional functions, double-precision constants and parameters all behave exactly
as expected.  If you find that they don't, then report it to us.  It's a flaw that we want
to fix.  We have found no technical reason that functions, parameters, and constants cannot
be made to mirror quite closely a mathematical language.

\noindent After the impulse function and the response function have been defined they
can be passed around to other routines that may be written to the abstract interface
of all functions, \verb#AbsFunction#.  This abstract interface allows one, essentially,
to evaluate the function using \verb#operator()(double x)# and to further compose it 
with functions, scalars, and parameters. In this example the functions \verb#impulse#
and \verb#response# are passed on to some plotters. The shape of the function is controlled
by the parameters; changing the vlaue of a parameter changes all the shapes of all functions
that depend on it, both primitive and derived.  In this example we have arranged things
so that the parameters are modified when the scale is moved, giving users a way to 
control the shapes of both functions while seeing the visual representation of the 
function respond in real time to the scale setting. 

\section{Automatic derivatives}

Any function, whether simple or composite, one dimensional or multidimensional,
can return partial derivatives with respect to any argument.  The computation is
done symbolically, in general, not numerically.  In case an analytic derivative
is not available for any function, a numerical calculation is provided instead.
For one-dimensional functions, the derivative is returned using this method:
\begin{verbatim}
  GENFUNCTION fprime = f.prime(); // F is a GENFUNCTION
\end{verbatim}
Partial derivatives of multidimensional functions with respect to one of their 
arguments is taken up in section~\ref{section:multidimensional} below.  Information 
on whether the derivative of a particular function is analytic or not can be
obtained through the use of the method \verb+bool AbsFunction::hasAnalyticDerivative() const+.

\section{The AbsFunction Class}

\noindent The abstract base class for all functions of one or more dimensions is 
\verb#AbsFunction#.  The header file for this class also defines operations 
on the class.  The base class describes the essential behaviour of all 
functions:
\begin{itemize}
\item{virtual double operator() (double) const = 0}
\item{virtual double operator() (const Argument \&) const = 0;}
\item{virtual unsigned int dimensionality() const;}
\item{virtual Derivative prime() const;}
\item{virtual bool hasAnalyticDerivative() const;}
\end{itemize}
It also defines a default constructor and a virtual destructor, and hides
the copy constructor and the default assignment operator.  The data type
\verb+Derivative+ is an \verb+AbsFunction+, and may be handled like any other
\verb+AbsFunction+.


\section {Multidimensional Functions and Arguments} 
\label{section:multidimensional}
\noindent Functions of more than one variable can be defined within this scheme.
The way this is handled is using the class \verb+Argument+, which essentially
is a vector-like list of double-precision numbers.  The argument is
constructed using the integer number-of-dimensions; then each element
of the argument can be set using \verb#operator[](int i)#; for example:
\begin {verbatim}
  Argument a(2);
  a[0]=1.0;
  a[1]=2.0;
  cout << myFunction(a) << endl;
\end{verbatim}

\noindent Users should take care to pass to functions an argument of the right
dimensionality.  Failure to do so will result in a run-time error. 
In addition care should be excercised when operating on functions with
binary operations like addition or multiplication, that the functions
both operands has the same dimensionality. Otherwise, a run-time error
results.

\noindent In case of functions of a one-dimensional argument, one may either call 
\begin{verbatim}
	operator() (double) const 
\end{verbatim}
or
\begin{verbatim}	
	operator() (const Argument &) const,
\end{verbatim}
and passing one-dimensional arguments only.  

For multidimensional functions, one can obtain the partial derivative with respect
to any argument using the method \verb+virtual Derivative AbsFunction::partial( unsigned int i) const+. An alternate way to express this depends upon our ability to associate names 
to components of a multidimensional argument, and is described in the next section.

\section {Variables}

The class \verb+Variable+ is the most elementary function.  The function value it returns
is just the argument itself.  It is useful because it can be used in function composition,
like this:
\begin{verbatim}
Sin sin;
Variable x;
GENFUNCTION f = sin(2.0*x);
\end{verbatim}

The class Variable can also be used to associate names with particular components of a
multidimensional argument, using the alternate form of the constructor, like this:
\begin{verbatim}
Variable x(0), y(1);
Genfunction f = cos(y)sin(x) + sin(x)cos(y);
\end{verbatim}
The function \verb+f+ is now a function of two variables.  Alternately, the function may
be constructed from the direct product operator \verb+%+, see section~\ref{section:arithmetic}
below.

If \verb+v+ is a variable and \verb+f+ is a multidimensional function, we can express
the partial derivative of \verb+f+ with respect to the variable \verb+v+ like this:
\begin{verbatim}
  GENFUNCTION fPartialV = f.partial(v); 
\end{verbatim}
i.e. using the method \verb+virtual Derivative AbsFunction::partial( const Variable & v) const+.


\section{AbsParameters}

\noindent \verb#AbsParameter# is an abstract base class for simple parameter
classes representing double precision numbers.  There are two main
types of \verb#AbsParameter#s.  The first kind, simply called 
\verb#Parameter#, comes with certain other services such as upper 
limits, lower limits, and connections to other parameters.  The other 
kind is composite parameters, resulting from arithmetic operations on 
parameters.  For now we will discuss only the first kind, \verb#Parameter#.  
Neither class is really intended to be extended any further by users.

\section{Parameters}

\noindent The first service that parameters perform is to provide limits: it is
impossible to set a parameter value outside of its limits--if this is
attempted the parameter value will be set to the maximum or minimum
allowed value.


\noindent Parameters can be made to take their value from other parameters. The
programmer sets this relationship through the \verb#connectFrom()# method, 
as illustrated here:
\begin{verbatim}
	Rectangular rect;
	Parameter height("Height", 1.0, 0.5, 2.0);
	rect.height().connectFrom(height);
\end{verbatim}

\noindent The first line creates a \verb#Rectangular# function.  The second line
instantiates a parameter object called \verb#height# with name "Height", lower
and upper limits of 0.5 and 2 and a value of 1.0; The third line first
accesses the internal parameter \verb#height()# used by the function and
connects this function permanently to the parameter \verb#height#.

\noindent Functions allow access to their parameters.  The shape of the function is 
then determined in one of three ways:

\begin{itemize}
\item {The default value of the parameter can be taken.}
\item {The value of the parameter can be set to a new value.}
\item {The value of the parameter can be obtained from another parameter 
(if the parameter is connected).}
\end{itemize}

\noindent When functions are used as arguments to arithmetic operations, and
their parameters are connected, the connections are maintained in the
composite function so created.  However if a parameter in a function
is connected after the function has been used in creating a composite,
the composite function does not respond to changes in the value of the
connected parameter.  Instead it has the value of the parameter at the
time it was created "frozen in.  

\subsection {Parameter class interface}

\noindent Here is the public interface to the parameter class; the destructor,
copy constructor , and assignment operator are not included on this
list but are defined; the default constructor is not defined.

\begin{verbatim}
Constructors:

  Parameter(std::string, 
            double, 
            double x0=-1e100, 
            double xl= 1e100);
  
  Accessors:

  const std::string & getName() const;
  double getValue() const;
  double getLowerLimit() const;
  double getUpperLimit() const;
  
  Modifiers:

  void setValue(double);
  void setLowerLimit(double lowerLimit);
  void setUpperLimit(double upperLimit);
  void connectFrom(Parameter *);

Print method: 

std::ostream & operator << ( std::ostream &, 
                             const Parameter &);

\end{verbatim}

\section {Arithmetic Operations}
\label{section:arithmetic}

\noindent The result of binary or unary arithmetic operation involving an
\verb#AbsFunction# is another \verb#AbsFunction#: however it is special kind of
\verb#AbsFunction#; generally the user will not care which kind since the
actual return type depends on the last operation to be evaluated in
the expression. The user needs to learn one trick, which is to handle
the result of the arithmetic operation through the base class.  This
is demonstrated here where, we multiply an exponential with a cosine:

\begin{verbatim}
1  Exponential   exp;             // makes an exponential
2  Cos           cosine;          // make a cosine
3  AbsFunction & f = exp*cosine;  // multiply the functions.
4  cout << f(5) << endl;          // evaluate & print. 
\end{verbatim}

\noindent The third line of this example is standard and safe.  It generates a
function of class \verb#FunctionProduct#, which also has type
\verb#AbsFunction#.  All operations on functions return
a value that has type \verb#AbsFunction#, although the class depends
on the actual operation, e.g. multiplication, division, et cetera.  However the user does not care about which
{\it class} is returned, but only cares that the return value has {\it
type} \verb#AbsFunction#, because he or she should handles it through the
base class, \verb#AbsFunction#.

\noindent Another way of accomplishing the same thing is provided by the data
type \verb#GENFUNCTION#, typedef'd back to \verb#const AbsFunction &#.  It allows
one to reexpress line 3 above in the following way:
\begin {verbatim}
GENFUNCTION f = exp*cos;
\end{verbatim}

\noindent You can use this syntax if you find it more natural. For
technically-minded people, the syntax generates a temporary object and
a reference thereto.  In ANSI standard C++ the temporary is guaranteed
to persist until the refererence \verb#f# goes out of scope.

\noindent Most of the operations that are valid on functions are also valid on
parameters.  Two parameters may be added, subtracted, multiplied, and
divided-to yield another parameter.  As in the case of functions, the
return type of an operation yielding a parameter is an \verb#AbsParameter &#,
which is typedef'd as \verb#GENPARAMETER#, for example:
\begin{verbatim}
	GENPARAMETER x_s = Delta_m/Gamma;
\end{verbatim}
where \verb#Delta_m# and \verb#Gamma# are both \verb#AbsParameters#.  Finally, operations
between \verb#AbsFunctions#, \verb#AbsParameters#, and simple double precision
numbers (we'll call them constants) are also defined.  Unary negation
is also defined on all of these data types.

\begin{table*}
\begin{center}
\begin{tabular}{|l|l|l|l|l|}
\hline
Symbol    & Name            & Operand 1   & Operand 2    & return type \\ \hline
 +,-,* /  & Simple          & AbsFunction & AbsFunction  & AbsFunction \\
          & Arithmetic      & AbsFunction & Constant     & AbsFunction \\
          & Operations      & Constant    & AbsFunction  & AbsFunction \\
	  &                 & AbsFunction & AbsParameter & AbsFunction \\
	  &                 & AbsParameter& AbsFunction  & AbsFunction \\
	  &                 & AbsParameter& AbsParameter & AbsParameter\\
	  &                 & Constant    & AbsParameter & AbsParameter\\
	  &                 & AbsParameter& Constant     & AbsParameter\\ \hline
 unary -  & Negation        & AbsFunction &              & AbsFunction \\ 
	  &                 & AbsParameter&              & AbsParameter\\ \hline
 ()       & Composition     & AbsFunction & AbsFunction  & AbsFunction \\ 
	  &                 & AbsFunction & AbsParameter & AbsParameter\\ \hline
 \%       & Direct Product  & AbsFunction & AbsFunction  & AbsFunction \\ \hline
convolve  & Convolution     & AbsFunction & AbsFunction  & AbsFunction \\ \hline
\end{tabular}
\end{center}
\end{table*}


\noindent Three additional operations are defined for functions.  The direct
product is for constructing multidimensional functions from
lower-dimensional functions.  For example a multivariate gaussian
distribution of two uncorrelated variables could be built from two one
dimensional gaussians using the following syntax:
\begin{verbatim}
Gaussian f;
Gaussian g;
GENFUNCTION h = f%g;
\end{verbatim}

\noindent Mathematically, this is equivalent to $f(x)*g(y) = h(x,y)$. Some
multidimensional functions may be constructed from lower dimensional
functions but of course not all functions can be constructed this way.
Where it fails, one can still resort to inheritance and build the
needed function by subclassing \verb#AbsFunction#. The picture in 
Fig.~\ref{ref:hydrogen} illustrates a multidimensional function obtained
by forming a three dimensional function from lower dimensional functions,
the illustration is the probability density function for a higher excited
state of hydrogen. 
\begin{figure}
\centerline{\makebox{\psfig{figure=hydrogen.ps}}}
\caption{Multidimensional functions can be expressed as a direct product
of lower dimensional functions; this example combines a Legendre Polynomial
with the product of an exponential, a power function and a Laguerre Polynomial
to obtain the probability density function of an electron in an excited state
of hydrogen.  The dots are generated by performing a random throwaway against
a generic function of three dimensions.}
\label{ref:hydrogen}
\end{figure}
 


\noindent The composition operator is for taking functions of functions, or
functions of parameters. It is denoted with the function call
operator, \verb#()#.  For example the following syntax is valid:

\begin {verbatim}
Exponential exponential;
Sin sine;
GENFUNCTION f = sine(exponential)	;
\end{verbatim}

\noindent A convolution of two functions has no special symbol-we ran out of
them- but is produced with the convolve method.  This function
performs a numerical convolution of the two functions treating the
first function as a response function, the second as a resolution
function.  The function has two addtional arguments, which are the
limits of the numerical convolution.  The convolve function is then
used like this:

\begin{verbatim}
Exponential exp;
Gaussian      gauss;
GENFUNCTION h = convolve(exp, gauss, -10.0, 10.0);
\end{verbatim}

\noindent The technique employed is to sample the product $f(x-y) g(y)$ at 200
points between $x0 < y < x1$, where $f$ is the response function and
$g$ is the resolution function.  We hope to use Fourier techniques to
improve this method in due time.

\section{Some Functions in the library}

\noindent Several types of functions have been implemented in this scheme.  A
table of those functions is shown here.  Certain functions, such as
Legendre Polynomials, take arguments to their constructor that specify
the order of the polynomial.  In other cases internal parameters 
govern the shape of the function.  The distinction is arbitrary at times, 
since, for example, Bessel functions need not be of integral order.
In general we have preferred to leave parameters out of functions if
one can obtain the same result by other means, for example, objects
of class \verb#Sin# do not have a frequency parameter since one
can obtain this as follows:
\begin{verbatim}
Parameter freq (``freq'', 10.0);	
Sin sine;
Variable x;	
GENFUNCTION f = sine(freq*x);	
\end{verbatim}

\begin{table*}
\begin{center}
\begin{tabular}{|l|l|l|}
\hline
Function                       & Name                              & Parameters              \\ \hline
Variable                       & Returns its own input             &                         \\ \hline
FixedConstant                  & Returns a constant                &                         \\ \hline
Sqrt                           & Sqrt                              &                         \\ \hline
Square                         & Returns square of input           &                         \\ \hline
Power                          & Returns a power of input          &                         \\ \hline
Exp                            & Exponential                       &                         \\ \hline
Sin                            & Sine                              &                         \\ \hline
Cos                            & Cosine                            &                         \\ \hline
Tan                            & Tangent                           &                         \\ \hline
Ln                             & Natural Logarithm                 &                         \\ \hline
Erf                            & Error function                    &                         \\ \hline
ForwardExp                     & Forward Exponentail tail          & decayConstant           \\ \hline
ReverseExp                     & Reverse Exponential tail          & decayConstant           \\ \hline
LogGamma                       & Natural log of Gamma function     &                         \\ \hline
IncompleteGamma                & Incomplete Gamma Function         & a                       \\ \hline
CumulativeChiSquare            & Probability($\chi^2$)              &                         \\ \hline                          
Gauss                          & Gaussian (Normal) distribution    & mean                    \\ 
                               &                                   & sigma                   \\ \hline
Landau                         & Landau distribution               & peak                    \\ 
                               &                                   & width                   \\ \hline
Rectangular                    & Rectangular function              & x0                      \\ 
                               &                                   & x1                      \\ 
                               &                                   & baseline                \\ 
                               &                                   & height                  \\ \hline
PeriodicRectangular            & Periodic rectangular              & spacing                 \\
                               &                                   & width                   \\ 
                               &                                   & height                  \\ \hline
SphericalBessel                & Spherical Bessel Functions        &                         \\ \hline
SphericalNeumann               & Spherical Neumann Function        &                         \\ \hline
AssociatedLaguerre             & Associated Laguerre Polynomial    &                         \\ \hline
AssociatedLegendre             & Associated Legendre Polynomial    &                         \\ \hline
AnalyticConvolution            & Moser-Roussarie convolutions      & frequency               \\       
                               &                                   & lifetime                \\       
                               &                                   & resolution              \\ \hline
IntegralOrder::Bessel          & Bessel and Neumann functions      &                         \\ \hline
FractionalOrder::Bessel        & Bessel and Neumann functions      & order                   \\ \hline
BivariateGaussian              & Gaussian in 2 variables           & mean0, mean1,           \\ 
                               &                                   & sigma0,sigma1, corr01   \\ \hline
TrivariateGaussian             & Gaussian in 3 variables           & mean[0-2],sigma[0-2]    \\
                               &                                   & corr[0-2][0-2]          \\ \hline
\end{tabular}
\end{center}
\end{table*}
\section {Parameter to Argument Adaptors}

Parameters control the shape of a function, which then act in the space of their
arguments.  In many applications it is useful to be able to create a new function
out of an existing function by treating one of its parameters as an argument.

A use case would be the following.  Suppose that one has written a function to
describe a measured quantity, like lifetime, smeared with a resolution function
which is a Gaussian.  The resulting function is parametrized by $\sigma$, the
width of the Gaussian distribution.  However to use this function in an unbinned
log likelihood fit, we often need to incorporate event-by-event estimates of 
$\sigma$.  To do this is to formally promote the parameter $\sigma$ to an argument,
in other words to turn:
\begin {equation}
	F(\alpha_i, \sigma; x)
\end {equation}
into
\begin{equation}
	F(\alpha_i; \sigma, x)
\end{equation}
We do this using a helper class called \verb+ParamToArgAdaptor+.  The class is an
\verb+AbsFunction+ that takes as arguments to its constructor:  
\begin{itemize}
\item{The name of an auxilary function, whose parameter it is to promote.}
\item{The name of a member function of the auxilary function, which accesses the 
parameter to promote.}
\end{itemize}

	
For example, one can turn a smeared expontial function of one variable into a
function of two variables, where the second variable is the event-per-event 
error estimate.
\begin{verbatim}
 Genfun::AnalyticConvolution  smearedExponential (Genfun::AnalyticConvolution::SMEARED_EXP);
 ParamToArgAdaptor<AnalyticConvolution> smExpEventByEvent(smearedExponential,
                                        &AnalyticConvolution::sigma);	
\end{verbatim}
The function \verb+smearedExponential+ is a function of one argument,
so the new function \verb+smExpEventByEvent+ is a function of 2
arguments.  The last argument is the event-by-event resolution.  Scale
factors may be added to the newly created function, so that the actual
value of the parameter used is some scale factor times the argument
provided.  The scale factor is in fact a parameter of the adaptor
class, accessed via the method
\verb+scaleFactor+, and may be set or connected just like any other parameter.

In case two parameters need to be promoted along these lines, we provide an additional
class:
\begin{verbatim}
DoubleParamToArgAdaptor
\end{verbatim}
which differs from \verb+ParamToArgAdaptor+ only by
the presence of an additional argument to the constructor (the name of a member function
accessing the second parameter to be promoted) and the presence of a second parameter
(an additional scale factor).

\section {How to extend the Generic Functions package with your own function}

\noindent Creating a function for the Generic Function package or inserting an
existing function into the framework has low work overhead.  Of course
if the function is complicated and hard to write, this the framework
does not make it easier!  However there are only a small number of
steps involved in creating a one- or multi-dimensional function and
endowing it with parameters that control its shape.  We present this
procedure as a checklist:

\begin{enumerate}
\item{Derive your function publically as a subclass of \verb#AbsFunction#.}

\item {What is the dimensionality of your function?  If the answer is "one", then you
don't need to override the method
\begin{verbatim}
  virtual unsigned int dimensionality() const;
\end{verbatim}
because the base class provides a default implementation which returns
the value 1.  Otherwise, if your function has two or more dimensions,
you will need to override the method.}

\item {Which parameters (if any) does your function depend on?  For
each parameter you need to add a \verb#Parameter# data member for that
parameter, and initialize it with its default value and range in the
constructor.  Also you will want a way of getting a reference to the
parameter-and actually you will want two methods, overloaded on const.
The purpose of having both methods is so that you will get back a
parameter that you can modify if the function itself is modifiable,
and a read-only parameter if the function is readonly.  If overloading
on const is strange to you, you can read up on it in Meyers; or you
can just adapt both of these example lines:
\begin{verbatim}
  Parameter & parm();
  const Parameter & parm() const;
\end{verbatim}
which shows, for example how to retrieve a parameter called "parm"
from a function. This is important and the compiler errors may appear
very mysterious if you don't write your parameter accessors this way.}

\item {You must provide an implementation of these two functions which 
are pure abstract in the base class:
\begin{verbatim}
  virtual double operator() (double) const;   
  virtual double operator() (const Argument &) const;
\end{verbatim}
These operators are what your function "does" so how you go about
doing this is your business.  But typically, for one dimensional
functions the second form of the function just calls the first; while
for multidimensional functions the first form generates a run-time
error and the second form checks the dimension of the argument before
evaluating it.}

\item { The composition operator, \verb#operator()( const AbsFunction &)# is 
overloaded in the base class and you don't want to hide it in the subclass,
so put the line \verb#using AbsFunction::operator()# in the header file
of your derived class.}

\item {Copy constructors are required for the proper operation of your
function within the framework.  You can use the compiler-generated
copy of this constructor if you wish; the usual caveats about
dynamically allocated memory apply here as with any class.  The
assignment operator is confusing because it only allows one to assign
Gaussians to Gaussians, exponentials to exponentials, et cetera, so
this function should be turned off.  The way to do this is to a)
declare the assignment operator, b) make it private, and c) do not
provide an implementation.}


\item {You will need to declare and define the method:
\begin{verbatim}
   virtual SubClass  * clone() const;
\end{verbatim}
where \verb#SubClass# is your new class.  This construction uses the
covariant return type mechanism, since the function in the base class
returns an \verb#AbsFunction *#.  The purpose of the routine is of course to
return a pointer to a newly allocated object.  The easiest way of
implementing this function is to use the copy constructor that you
wrote (if you wrote one, otherwise you can take the default copy
constructor that the compiler wrote for you).}

\item {{\bf Optional} If you will be providing an analytic derivative for your
new function, then override the method
\begin{verbatim}	
  Derivative partial (unsigned int) const;	
\end{verbatim}

The function \verb+Sin+ provides a good example of how to do this.  Should you
choose to provide an analytic derivative, you should also then override the
method
\begin{verbatim}
 bool hasAnalyticDerivative() const 
\end{verbatim}
so that it returns true.  Our preference is to put the implementation right in
the header file since it provides useful information (``yes, this class has an
analtyic derivative'') for users. }
\end{enumerate}


\section {Cut classes}

GenericFunctions is based on our ability, in C++, to write classes
having the algebra of real functions of one or more variables.  The
same techniques which have been applied here can be used to describe
other algebras.  In particular, there is an extremely useful extension
to {\it cut classes}, which are objects that return a true or false
decision on some data type, and which support the algebra of boolean
operations.  A set of base classes has been written to allow one to
create cuts on any data type, to combine them using the operators
\verb+||+, \verb+&&+ and!  We have written these classes to
interoperate with with the standard template library.

A simple example will show the usefulness of this.  In this example, the cut-object
\verb+IsPrime+ and \verb+IsInRange+ have been written as subclasses of \verb+Cut<int>+.  In 
the example we use these cuts to select prime numbers in the range 30-60, and print
out the result (31,37, 41, 43, 47, 53, 59) to the terminal screen:
\begin{verbatim}
int main(int, char **) {
  //
  // Make an array of integers:
  //
  const int LENGTH=100;
  int integers[LENGTH];
  //
  // Fill them with the integers:
  //
  for (int i=0;i<LENGTH;i++) integers[i]=i;
  //
  // Make an output iterator:
  //
  std::ostream_iterator<int> dest(std::cout,"\n");
  //
  // Cut on prime numbers between 30 and 60:
  //
  const Cut<int>::Predicate cut = IsPrime() && IsInRange(30, 60);
  std::remove_copy_if(integers, integers+LENGTH, dest, !cut);
  //
  // Bye:
  //
  return 0;
}
\end{verbatim}

Let's look in detail at how one of these cut objects was declared 
(note that it inherits from \verb+Cut<int>+):
\begin{verbatim}
class IsPrime:public Cut<int> {

public:

  // Constructor:
  IsPrime();

  // Destructor:
  virtual ~IsPrime();

  // Truth operator:
  bool operator () (const int & arg) const;

  // Clone
  virtual IsPrime *clone() const;

};

\end{verbatim}

Readers familiar with the standard template library will ask, ``How does
this relate to STL predicate types?  What has been gained, and what has 
been the cost?''  One has gained the ability to combine existing cut-objects
with boolean operations.  The cost has been:  some virtual function calls
in the evaluation of these objects, and the necessity of inheriting from 
\verb+Cut<Type>+, meaning also that two pure virtual functions:
\begin{itemize}
\item{\verb+virtual Cut * clone() const = 0;+   }
\item{\verb+virtual bool operator ()( const Type & t ) const = 0;+}
\end{itemize}
will have to be defined by the user.  

Unlike the function-objects, we do not provide any concrete cut-objects.  Instead
what we provide is the base classes that allow one to easily write STL predicates
supporting boolean operations\footnote{To be sure, the STL has a set of classes that
allow one to combine cuts using boolean operations, but their interface
is appalling.} To access these, include the header file \verb+CLHEP/GenericFunctions/TrackCutBase.hh+.
\end{document}
