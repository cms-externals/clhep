% this is latex
\documentclass{article}
\usepackage{html}

\setlength{\textheight}{230mm}
\setlength{\topmargin}{0mm}

%define page size:
\setlength{\headheight}{0mm}
\setlength{\headsep}{0mm}
\setlength{\footskip}{12mm}
\setlength{\textwidth}{160mm}
\setlength{\oddsidemargin}{0mm}
\setlength{\evensidemargin}{0mm}

\newcommand{\lbar}
           {\makebox[0mm][l]{\hspace{.1em}\rule[.9ex]{.4em}{.4pt}}\lambda}

\begin{document}

\section*{CLHEP Units}

The {\it CLHEP Units} module has been supplied by \latexonly{\tt GEANT4.}
\begin{rawhtml}
<A target="_top" href="http://wwwinfo.cern.ch/asd/geant4/geant4.html">
<TT>GEANT4</TT></a>.
\end{rawhtml}
It consists of two header files which contain definitions of some frequently
used physical constants and units:
\begin{latexonly}
\begin{verbatim}
   CLHEP/Units/SystemOfUnits.h
   CLHEP/Units/PhysicalConstants.h
\end{verbatim}
\end{latexonly}
\begin{rawhtml}
<PRE>
   <A HREF="../../RefGuide/Units/SystemOfUnits_h.html"><TT>CLHEP/Units/SystemOfUnits.h</TT></A>
   <A HREF="../../RefGuide/Units/PhysicalConstants_h.html"><TT>CLHEP/Units/PhysicalConstants.h</TT></A>
</PRE>
\end{rawhtml}

To make them available it is enough to insert in your program the following
line:
\begin{verbatim}
   #include "CLHEP/Units/PhysicalConstants.h"
\end{verbatim}

All constants and units are defined via few so called {\it basic} units.
The following units have been choosen as {\it basic}:
\begin{itemize}
\item {\it millimeter} for legth
\item {\it nanosecond} for time
\item {\it MeV} for energy
\item {\it positon charge} for electric charge
\item {\it Kelvin} for temperature
\item {\it mole} for amount of substance 
\item {\it radian} for plane angles
\item {\it steradian} for solid angles
\end{itemize}

The {\it CLHEP Units} module can be considered as an attempt to provide
a practical System of Units for HEP applications.
Many standard HEP classes, for example in {\tt GEANT4} and {\tt CLHEP},
assume that data are given in the System of Units defined in the
{\it CLHEP Units} module. For this reason it is recommended to define any
physical data with its units, e.g.
\begin{verbatim}
   crossection = 3.5 * barn
   density     = 10. * g/cm3
\end{verbatim}

Tables \ref{tab1} and \ref{tab2} represent physical units and
physical constants defined in the {\it CLHEP Units} module.
Most of the physical constants were initially taken from
the Particle Data Book: {\it "Phys. Rev. D volume 50 3-1 (1994) page 1233"}.
As of release 1.9.4.1/2.0.4.1, the constants have been updated to reflect 
the 2008 PDG values: {\it "Physics Letters B667 (2008) page 103"}.

\begin{table}
\centering
\begin{tabular}{|l|l|l|l|}
\hline
\bf{Physical quantity}&
\bf{{\it CLHEP Units} name}&
\bf{Name of unit}&\bf{Symbol, equation}\\
\hline
Length, area, volume&{\tt mm,mm2,mm3}    & millimeter   &$ mm,mm^2,mm^3  $\\
                   &{\tt cm,cm2,cm3}    & centimeter   &$ cm,cm^2,cm^3   $\\
                   &{\tt m,m2,m3}       & meter        &$ m,m^2,m^3      $\\ 
                   &{\tt km,km2,km3}    & kilometer    &$ km,km^2,km^3   $\\
                   &{\tt parsec}        & &$pc=3.0856775807\times 10^{16}\,m$\\
                   &{\tt microm}        & micro meter  &$                $\\
                   &{\tt nanom}         & nano meter   &$                $\\
                   &{\tt fermi}         &              &$ 10^{-15}\,m    $\\
                   &{\tt barn}          &              &$ 10^{-28}\,m^2  $\\ 
                   &{\tt millibarn}     &              &$                $\\
                   &{\tt microbarn}     &              &$                $\\
                   &{\tt nanobarn}      &              &$                $\\
Angle              &{\tt rad}           & radian       &$ rad            $\\
                   &{\tt mrad}          & milli radian &$                $\\
                   &{\tt deg}           & degree       &$ (\pi/180)\,rad $\\
                   &{\tt st}            & steradian    &$ sr             $\\
Time               &{\tt s}             & second       &$ s              $\\
                   &{\tt ms}            & milli second &$ ms             $\\
                   &{\tt ns}            & nano second  &$ ns             $\\
Frequency          &{\tt Hz,kHz,MHz}    & hertz        &$ Hz,kHz,MHz     $\\
Energy             &{\tt eV,keV,MeV,GeV,TeV}
                                        & electron volt&$eV,keV,MeV,GeV,TeV$\\
                   &{\tt joule}         & &$ J=6.24150\times 10^{12}\,MeV$\\
Mass               &{\tt kg}            & kilogram     &$ kg=J\,s^2/m^2  $\\
                   &{\tt g}             & gram         &$ g              $\\ 
                   &{\tt mg}            & milli gram   &$ mg             $\\ 
Force              &{\tt newton}        &              &$ N              $\\
Power              &{\tt watt}          &              &$ W              $\\
Pressure           &{\tt pascal}        & pascal       &$ Pa             $\\
                   &{\tt bar}           &              &$ 10^5 \, Pa     $\\
                   &{\tt atmosphere}    & &$ 1.01325\times 10^5\,Pa      $\\
Electric charge    &{\tt eplus}         &positon charge&$ e              $\\
                   &{\tt coulomb}       & &$ C=6.24150\times10^{18}\,e   $\\
Electric current   &{\tt ampere}        &              &$ A              $\\
Electric potential &{\tt volt}          &              &$ V              $\\
                   &{\tt kilovolt}      &              &$ kV             $\\
                   &{\tt Megavolt}      &              &$ MV             $\\
Electric resistence&{\tt ohm}           &              &$ \Omega         $\\
Electric capacitance&{\tt farad}        &              &$ F              $\\
                   &{\tt millifarad}    &              &$ mF             $\\
                   &{\tt microfarad}    &              &$ \mu F          $\\
                   &{\tt nanofarad}     &              &$ nF             $\\
                   &{\tt picofarad}     &              &$ pF             $\\
Magnetic flux      &{\tt weber}         &              &$ Wb             $\\
Magnetic field     &{\tt tesla}         &              &$ T              $\\
                   &{\tt gauss}         &              &$ G=10^{-4}\,T   $\\
                   &{\tt kilogauss}     &              &$ kG             $\\
Inductance         &{\tt henry}         &              &$ H              $\\
Temperature        &{\tt kelvin}        &              &$ K              $\\
Amount of substance&{\tt mole}          &              &$ mol            $\\
Activity           &{\tt becquerel}     &              &$ Bq             $\\
                   &{\tt curie}         & &$ 3.7\times10^{10}\,Bq        $\\
Absorbed Dose      &{\tt gray}          &              &$ Gy             $\\
\hline
\end {tabular}
\caption{Physical units defined in the {\it CLHEP Units} module}
\label{tab1}
\end{table}

\begin{table}
\centering
\begin{tabular}{|l|l|l|}
\hline
\bf{Physical quantity}    &\bf{{\it CLHEP Units} name}&\bf{Symbol, equation}  \\
\hline
positon charge in coulomb &{\tt e\_{}SI}         &$ 1.602176487\times10^{-19}$\\
speed of light in vacuum  &{\tt c\_{}light}      &$ c                      $\\
                          &{\tt c\_{}squared}    &$ c^2                    $\\
Plank constant            &{\tt h\_{}Planck}     &$ h                      $\\
Plank constant, reduced   &{\tt hbar\_{}Planck}  &$ \hbar                  $\\
                          &{\tt hbarc}           &$ \hbar\,c               $\\
                          &{\tt hbarc\_{}squared}&$ (\hbar\,c)^2           $\\
electron charge           &{\tt electron\_{}charge}&$ -e                   $\\
                          &{\tt e\_{}squared}    &$ e^2                    $\\
atomic equivalent mass unit&{\tt amu\_{}c2}      &$ 931.494028\,MeV        $\\
atomic mass unit          &{\tt amu}             &                          \\
electron mass             &{\tt electron\_{}mass\_{}c2}&$ m_e\,c^2         $\\
proton mass               &{\tt proton\_{}mass\_{}c2}  &$ m_p\,c^2         $\\
neutron mass              &{\tt neutron\_{}mass\_{}c2} &$ m_n\,c^2         $\\
permeability of free space&{\tt mu0}             &$ \mu_0                  $\\
permittivity of free space&{\tt epsilon0}        &$ \epsilon_0             $\\
electromagnetic coupling  &{\tt elm\_{}coupling} &$ e^2/4\pi\epsilon_0     $\\
fine-structure constant   &{\tt fine\_{}structure\_{}const} &$ \alpha      $\\
classical electron radius &{\tt classic\_{}electr\_{}radius}&$  r_e        $\\
electron Compton wavelength&{\tt electron\_{}Compton\_{}length}&$ \lbar_e  $\\
Bohr\_{}radius            &{\tt Bohr\_{}radius}  &$ a_\infty               $\\
                          &{\tt alpha\_{}rcl2}   &$ \alpha\,r_e^2          $\\
                          &{\tt twopi\_{}mc2\_{}rcl2}&$ 2\pi\,m_e\,c^2\,r_e^2$\\
Avogadro constant         &{\tt Avogadro}        &$ N_A                    $\\
Boltzmann constant        &{\tt k\_Boltzmann}    &$ k                      $\\
                          &{\tt STP\_{}Temperature}&$ 273.15\,K            $\\
                          &{\tt STP\_{}Pressure} &$ 1\,atmosphere          $\\
                          &{\tt kGasThreshold}   &$ 10^{-2}\,g/cm^3        $\\
\hline
                          &{\tt pi}              &$ \pi                    $\\
                          &{\tt twopi}           &$ 2\pi                   $\\
                          &{\tt halfpi}          &$ \pi/2                  $\\
                          &{\tt pi2}             &$ \pi^2                  $\\
                          &{\tt perCent}         &$ 10^{-2}                $\\
                          &{\tt perThousand}     &$ 10^{-3}                $\\
                          &{\tt perMillion}      &$ 10^{-6}                $\\
\hline 
\end {tabular}
\caption{Physical constants defined in the {\it CLHEP Units} module}
\label{tab2}
\end{table}

\end{document}
